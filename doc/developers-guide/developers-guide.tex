\documentclass[11pt, oneside]{scrartcl}
\usepackage{hyperref}
\usepackage{mdwlist}  % for reduced space enumerations, itemizations (use {enumerate*} and {itemize*})

% Configure Hyperlinks appeareance
\hypersetup
{
  %draft = true,          % Shut off all hyperlinks
  colorlinks=true,        % Set off boxes around links (color text)
  linkcolor=blue,
  citecolor=blue,
  urlcolor=blue
}
%\urlstyle{rm}                   % Set www links text to default text
\usepackage{breakurl}

\newcommand{\backtocontents}{\hyperref[contents]{Back to Contents.\ref*{contents}}}

\title{\Large{preCICE} \\[5mm] \Huge{Developer's Guide}}

%\author{by Bernhard Gatzhammer}


\begin{document}
  %\setlength{\parskip}{2ex}

\maketitle
\tableofcontents \label{contents}

% =============================================================================

\section{Introduction} 

This guide is meant for all people aiming at working on the source code of
preCICE. In section~\ref{chap:setup} a development environment based on Eclipse
is configured. It is not necessary to work with Eclipse, but Eclipse has proven
to be a valuable tool, especially with regards to its plugin capabilities. One
very important plugin simplifying SVN operations is Subclipse, whose basics are
explained in section~\ref{chap:subclipse}. section~\ref{chap:prerequesits} lists
basic C++ programming and software pattern knowledge that should be clear before
starting programming on preCICE, otherwise no source code of good quality can
result. All coding conventions and the coding style is given in
section~\ref{chap:styleguide}. Every programmer working on preCICE should stay
as close as possible to these rules, unless there are serious reasons not to do
so.

\backtocontents

% =============================================================================

\section{Setup} \label{chap:setup}

This section describes how to setup a development environment for preCICE using
Eclipse for coding and version control (with Subversion, SVN), and using Eclipse
or SCons for building libraries and executables of preCICE.

\backtocontents    
    
% -----------------------------------------------------------------------------
    
\subsection{Preparing Eclipse}
   
Eclipse can be downloaded for free under \url{http://www.eclipse.org/}. In order
to program C++ in Eclipse, you should use the CDT plugin. You can get Eclipse as
bundle with CDT, or get CDT afterwards as plugin. It is highly recommended to
work with the plugin Subclipse that interfaces Subversion (see
section~\ref{chap:subclipse}). When working with Subclipse under Windows, make
sure you also have downloaded and selected the SVNKit as client in the SVN
interface to SVN (in preferences$\rightarrow$team$\rightarrow$SVN), in order to 
avoid unnecessary repeated password typing.

Plugins are installed into Eclipse as follows:
\begin{enumerate*}
\item Click on the "Help" menu in Eclipse
\item Choose the "Install new software" entry
\item Enter the address of the new plugin into the ``Work with:`` field
\item Wait until the available modules of the plugin are displayed
\item Select the modules to be installed (usually all)
\item Continue by clicking ``Next" and perform remaining installation steps
\end{enumerate*}

\noindent The plugin address of the CDT can be found here
\url{http://www.eclipse.org/cdt/downloads.php}. The subclipse plugin address can
be found at \url{http://subclipse.tigris.org/} under downloads.
   
\backtocontents   
   
% -----------------------------------------------------------------------------
   
\subsection{Eclipse Basics}
    
Working with Eclipse requires to define one (or several) \textit{workspaces},
where the source code of the C++ projects (and other projects) is located. When
starting Eclipse for the first time, the user needs to declare a directory to be
a workspace. Manipulations of files within that workspace should be done through
Eclipse only, in order to avoid problems. Eclipse creates special workspace and
project files starting with a dot, that contain information about the projects
and files contained in a workspace. These files are adapted automatically when
files are manipulated through the Eclipse workbench. Nevertheless, in some cases
it becomes necessary to do changes in the workspace outside of Eclipse. In order
to make Eclipse recognize these changes, you have to refresh the ProjectExplorer
view (by selecting the corresponding workspace folders in the ProjectExplorer
and pressing F5, e.g.).

The content of the Eclipse workbench is configured through
\textit{perspectives}, which are tailored to the different tasks a user wants to
do. Important perspectives are the C/C++, SVN Repository Exploring, and Team
Synchronization perspectives. A perspective can be customized to show different
\textit{views}, which are subtags offering different functionalities.

A new C++ project can be created by clicking on the "File" menu, then at the
entry "New" and selecting "C++ Project".
   
\backtocontents   
   
% -----------------------------------------------------------------------------

\subsection{Retrieving Source Code from Subversion Repository with Eclipse} 
\label{sec:setup_retrieve_sources}
    
To retrieve the sources of preCICE (and the related project Peano) the Subclipse
plugin of Eclipse can be used. The source codes are kept in a \textit{version
control system} called Subversion (SVN), which enables to retrieve any old
version of the sources and lets several people work on the same sources
concurrently in a copy-change-merge workflow. Detailed descriptions of SVN can
be found under \url{http://svnbook.red-bean.com/nightly/en/index.html}.
Subclipse is a graphical interface to SVN in Eclipse and enables simple and fast
SVN operations. Documentation for Subclipse can be found at
\url{http://help.collab.net/index.jsp?topic=/org.tigris.subclipse.doc/topics/toc.html}.
The first step to get the sources of preCICE is to setup the location
information of the SVN repository in the "SVN Repository Exploring" perspective.
Before performing the steps listed below, make sure that you are in the user
groups enabling the work with the preCICE and Peano repositories (i.e.
\texttt{svn\_precice} and \texttt{svn\_peano}) by using the command
\\[2mm] 
\verb|   $ groups | 
\\[2mm]
and checking whether the required groups are listed.
\\[3mm] 

\textbf{ATTENTION:} Be careful in the "SVN Repository Exploring" perspective, it
is possible to delete sources from the SVN repository and all its old versions.
\\[3mm]

The repository locations of preCICE and Peano are added as follows:
\begin{enumerate*}
\item Choose the "SVN Repository Exploring" perspective in Eclipse
\item Activate the "SVN Repositories" view
\item Rightclick at an empty spot in the "SVN Repositories" view
\item Select "New" and then "Repository Location..."
\item Enter the URL (you have to adapt the USERNAME, the machine number XY,
and replace \{at\} by its symbol):
\url{svn+ssh://USERNAME{at}atsccsXY.informatik.tu-muenchen.de/home_local/repositories/svn/precice}
\\ $\rightarrow$ an entry for the preCICE repository appears in the "SVN Repositories" view
\item Add the repository location for peano, which has the same location
but ends with "peano" instead of "precice"
\end{enumerate*}

When expanding the repository location of preCICE or Peano, three subfolders
appear:
\begin{itemize*}
\item branches
\item tags
\item trunk
\end{itemize*}

All of the three subfolders do contain complete versions of a project, but only
the \textit{trunk} has the most current version of the project. Thus, only the
resources contained in this folder are necessary for development work on
preCICE. Development does not take place directly on the resources in the SVN
repository, but requires a \textit{checkout} of the resources to a different
place, which will be the before created workspace of Eclipse. Perform the
following procedure for the preCICE \textbf{and} Peano repository:

\begin{enumerate*}
\item Expand the repository location in the "SVN Repositories" view in Eclipse
\item Rightclick on the trunk folder
\item Select "Checkout..."
\item Make sure that the radio button "Check out as a project using the New
Project Wizard" is selected
\item Make sure that the checkbox "Check out HEAD revision" is selected
\item Click on the "Finish" button and create a C++ project in the workspace
\end{enumerate*}
The checkout may take several minutes, depending on your network connection. The
Peano repository is quite big, it is sufficient and recommended to checkout only
the directory \verb$src$ of Peano.

\backtocontents

% -----------------------------------------------------------------------------
   
\subsection{Retrieving external tools and libraries needed to build preCICE}
 
preCICE needs other external sources and libraries to be built succesfully. In
addition to the preCICE and Peano sources, the following projects have to be
available
\begin{itemize*}
\item Boost C++ libraries (no installation is necessary) from \url{http://www.boost.org/}. Check which version is compatible in Downloads.
\item MPI (recommended and mandatory for MPI socket communication is the newest
MPICH2 version) from \url{http://www.mcs.anl.gov/research/projects/mpich2/} (a
version of MPI2 is probably installed already)

\end{itemize*}

\backtocontents
  
% -----------------------------------------------------------------------------  
      
\subsection{Building preCICE}
    
preCICE is built using the software construction tool SCons, which is based on
Python. Hence, Python (version smaller than 3.0) is required to install SCons.
Python can be downloaded from \url{http://www.python.org/download/}. SCons can
be downloaded from \url{http://www.scons.org/}. For installation instructions
look in the documentation of Python and SCons respectively. Please note, Python
is probably already installed on the machine you are working on as may be SCons.
If you do not have root rights on the machine you want to install Python or
SCons, you have to do use the \texttt{prefix} argument on configuration of the
programs to install them at a location you have full rights. SConstruct files
are the root build files, and describe what has to be built and how. The preCICE
project containes an SConstruct file as well as the Peano project. Actually, in
the preCICE project there is one file for building under Windows,
\texttt{SConstruct-windows}, and one for building under Linux,
\texttt{SConstruct-linux}.

Before building preCICE, a couple of environment variables have to be exported:
\begin{itemize*}
\item Export the environment variable \texttt{TARCH\_SRC} to be the absolute
path to the Peano \verb$src$ directory.
\item Export the environment variable \texttt{BOOST\_ROOT} to be the absolute
path to the Boost project root directory.
\end{itemize*}
To export an environment variable the following command can be used: 
\\[2mm]
\verb|   $ export VARIABLE_NAME=VARIABLE_VALUE |
\\[2mm]
The content of a variable can be checked by
\\[2mm]
\verb|   $ echo $VARIABLE_NAME |
\\[2mm]
In order to save the time of repeated exports, the export commands should be
added to the \texttt{.bashrc} (for bash command shell) file in the home
directory of a user. After adding the commands to the \texttt{.bashrc} file, a
command shell must be reopened or the command
\\[2mm]
\verb|   $ source .bashrc |
\\[2mm]
has to be issued to register the changes made to \texttt{.bashrc}.        
\\[3mm]
\textbf{ATTENTION:} Be careful with changing the \texttt{.bashrc} file, if you
introduce errors you might not be able to login to your user account any more.
\\[3mm]
In Windows, environment variables can be set in the extended system control menu.

\backtocontents

\subsubsection{Building from Command Line}

For building preCICE on Linux change to its project root directory and type
\\[2mm] \verb|   $ scons -f SConstruct-linux| \\[2mm] For building preCICE on
Windows type \\[2mm] \verb|   $ scons -f SConstruct-windows| \\[2mm] On a
machine with several cores, you can use more than one core by adding the
\texttt{-j2} option (for 2 cores, e.g.) when invoking the build.
An executable and a library of preCICE are then built into the folder \\
\texttt{precice\_root/build/debug-dim2-mpi-serial/}.

\backtocontents
        
\subsubsection{Building from Eclipse}
      
Under Linux, SCons builds can be integrated into Eclipse CDT, which prevents one
from using the command line and allows to highlight compile errors directly in
the source files. In order to enable the use of SCons in Eclipse, the following
steps have to be done:
\begin{enumerate*}
\item Open the project properties (by rightclicking at the project root
folder in the ``Project Explorer'' view and selecting the corresponding
menu point, e.g.)
\item Select the ``C/C++ Build'' entry from the left menu pane
\item Select the tab ``Builder Settings'' on the right window frame, which shows
the settings for the currently activated build configuration (which is displayed
in the top of the window)
\item Switch off the checkbox ``Use default build command'' and type instead the
desired SCons command (\texttt{scons -f SConstruct-linux build=debug}, e.g.);
one Eclipse build configuration can trigger one SCons build configuration, thus. 
\item Switch off the checkbox ``Generate makefiles automatically''
\item Select the root path of the project as build location
\item Switch to the tab ``Behaviour'' then, to further adjust build settings
\item Delete the string in the textbox ``Build (incremental build)''
\item Replace the string in the ``clean'' textbox by \texttt{-c}, which is the
SCons clean argument
\end{enumerate*}
Now, you should be able to build with SCons in Eclipse. Errors, warnings and
info messages appearing as build output in the ``Console'' view are collected in
the ``Problems'' view and additionally highlighted in the source code files and
in the ``Project Explorer'' view of Eclipse.        
   
\backtocontents   
      
% \subsubsection{Building with Eclipse CDT internal build tools}
%       
% Although the main way to build preCICE is with SCons, the Eclipse C++ plugin CDT
% allows to automatically create makefiles. Building with Eclipse makefiles
% requires to setup all build options in a graphical menu. The procedure can be
% done as follows:
% \begin{enumerate*}
% \item Rightclick at the project root folder in the "Project Explorer" view in
% Eclipse. It is important to \textit{\textbf{always} click on the root folder}
% of the project. Otherwise the changes do not affect the whole project and it is
% hard or impossible to switch back to the settings of the whole project!
% \item Select "Project Properties"
% \item Select and expand "C/C++ Build"
% \item Select "Settings"
% \item In the tab "Build Artifact", select whether to build a library or an
% executable and choose the name of the artifact to be built
% \item In the tab "Tool Settings" adapt the entries of the "C++ Compiler" entry
% and the linker (available only when building an executable).
% \end{enumerate*}
%         
% The necessary settings can be deduced from the Scons build output or buildfiles.
% Eclipse offers the possibility to have several build configurations with
% individual settings, which speeds up switching between different builds a lot.
% In order to use necessary components of Peano in preCICE, the following steps
% have to be done:
% \begin{enumerate*}
% \item Ensure that both Peano and preCICE act on the same compile flags
% \item Add Peano's src directory to preCICE's build path
% \item Ensure that all of Peano's components besides 
% \begin{itemize*}
% \item \texttt{blitz}
% \item \texttt{configuration}
% \item \texttt{irr}
% \item \texttt{plotter}
% \item \texttt{utils}
% \end{itemize*}
% are excluded from the Peano build, by rightclicking on the corresponding
% component in the \texttt{src} directory of Peano and selecting "Exclude from build..."
% \item Adapt the file \texttt{Components.h} in the \texttt{src} directory of
% Peano to include only those components listed above
% \item Change the build artifact for Peano to static library
% \item Add the name and location of the Peano library built to the linker build
% settings of the preCICE exectuble
% \end{enumerate*}
   
% =============================================================================
   
\section{SVN with Subclipse} \label{chap:subclipse}
 

% -----------------------------------------------------------------------------
  
\subsection{How to operate SVN with Subclipse}

As described in section~\ref{chap:setup}, Subclipse is a graphical frontend for
SVN. This section describes the basic operations and terms needed to operate
with SVN in Eclipse.

Operations and terms discussed are
\begin{itemize*}
\item Checkout
\item Update
\item Commit
\item Conflicts
\item Team synchronization
\item Compare view
\end{itemize*}

\backtocontents

% -----------------------------------------------------------------------------

\subsection{Checkout}

A checkout is done to retrieve resources from an SVN repository. The procedure
is described in detail in subsection~\ref{sec:setup_retrieve_sources}. The checked
out resources are a copy of the resources in the repository, i.e. changing the
checked out resources does not affect the resources in the repository. This is
the intended way to work on the resources of the repository.
\\[3mm]

\noindent
\textbf{ATTENTION:} Do \textit{never} work directly on the
resources of a repository!!

\backtocontents

% -----------------------------------------------------------------------------

\subsection{Commit}

In order to apply the local changes of the repository resources to the resources
of the repository itself, a \textit{commit} has to be done. Before the commit,
the local changes are not seen by any other developers. Afterwards the other
developers can retrieve the modified resources via an update (see
subsection~\ref{sec:subclipse-update}). The commit operation should always be done
in the "Team Synchronizing" perspective, which is described in
subsection~\ref{sec:subclipse-teamsync}. When performing a commit, a non-empty
message \textbf{has to be} attached, otherwise the commit fails. This message
should contain a brief description of the changes done to the resources to be
committed.
  
\backtocontents  
    
% -----------------------------------------------------------------------------
    
\subsection{Update} \label{sec:subclipse-update}

Since several people can work concurrently on the resources of a repository. The
changes done by other developers must be somehow transported into the own local
copy. The operation performed to get these changes is called \textit{update}. In
Eclipse, an update should always be performed in the "Team Synchronizing"
perspective, which is described in subsection~\ref{sec:subclipse-teamsync}.

\backtocontents

% -----------------------------------------------------------------------------

\subsection{Conflicts}

SVN allows several users to work on the same resource (file) of a repository.
Thus, it is possible that two developers do change exactly the same part of a
file in their local copy. The first developer commiting his changes will not run
into any difficulties. However, the second developer encounters a special
situation when trying to update or commit the very same resource, which is
called a \textit{conflict}. A conflict can only be resolved by comparing the
version of the resource in the repository with the local version of the
resource. The "Team Synchronization" pesrpective provides functionalities to
resolve such conflicts and is described in subsection~\ref{sec:subclipse-teamsync}.

\backtocontents

% -----------------------------------------------------------------------------

\subsection{Team Synchronization} \label{sec:subclipse-teamsync}

Subclipse allows a direct commit of resources in the "Project Explorer" view.
However, if a commit of a resource having conflicts is done, the conflicting
parts will be overwritten by the local changes without any notice. This means,
the work of another developer will be overwritten! Thus, never directly commit
your local changes into the repository.

A similar situation can occur when performing a direct update of resources.
Conflicting parts of resources are then enhanced with the version from the
repository and marked by special character sequences (\verb|>>>>>>>>>>>>>|,
e.g.). The developer has then to go through the resources and find the places
where conflicts have occured. This can be time consuming and possibly introduce
errors in the code. Hence, it should be avoided whenever possible.

To improve this situation, the "Team Synchronization" perspective offers a
solution. The general procedure for performing updates and commits should be as
follows:
\begin{enumerate*}
\item Select the resources that sould be updated/commited in the "Project
Explorer" view of Eclipse.
\item Rightclick on one of the selected resources and select "Team" and
"Synchronize with Repository". \\ $\rightarrow$ the perspective changes to
"Team Synchronization"
\item Wait until the progress bar (in the lower right) has completed.
\item Open (if not opened) the "Synchronize" view.
\item Select the button "Conflicts Mode" within the "Synchronize" view \\
$\rightarrow$ all conflicting resources are shown.
\item Resolve all conflicts of files that should be updated or commited.
\item Click on the button "Incoming Mode" to see possible updates of resources.
\item Perform all desired updates.
\item Click on the button "Outgoing Mode" to see possible commits of resources.
\item Before doing any commits, \textit{make sure that the version of the code
you posess is functional}. This is especially important, when you have
retrieved updates before or if you have resolved conflicts. Other developers
might use different compile flags and possibly have overlooked errors in the
code that you will discover when compiling or running test cases. Resolving
conflicts is error-prone, even with the help of Subclipse, and makes checking
the code a necessary task before any commit. After you have checked the
correctness of the code, perform all desired commits.
\end{enumerate*}
Updates and commits can be simply done by selecting the resources in the
"Synchronize" view, rightclicking on one of the selected resources and
selecting "Update" or "Commit...". How to solve conflicts and to check what
parts of the resources will be updated or commited is explained in the next
subsection.

\backtocontents

% -----------------------------------------------------------------------------

\subsection{Compare view}

In order to see the conflicts, updates and commits of a resource, double click
on the resource. A compare view opens, which shows the local version of the
resource on the left hand side, and the repository version on the right hand
side. Conflicting parts are highlighted in red, updates in blue, and commits in
grey. By clicking the button "Copy all non-conflicting parts from right to
left", all updates will be retrieved. Conflicts and commits are remaining. When
clicking on the box of a connecting line of two highlighted code snippets (for
updates and conflicts), the version of the repository will replace the local one
in case of an update. In case of a conflict, the version of the repository will
be \textit{added} to that of the local version. This means, that in case of a
conflict, careful by-hand copying of the parts to be kept is necessary. After
you have modified your local version of a resource in the compare view, don't
forget to save the resource. If you have resolved all conflicts of a file to
your satisfaction (this does not necessarily mean, that the compare view does
not show any more conflicts), rightclick on the resource in the "Synchronize"
view and select "Mark as Merged". The file will then go into the category
"commit" or "update".

\backtocontents
      
% =============================================================================

\section{Programming Prerequesits} \label{chap:prerequesits}
 
The following concepts are employed in the source code of preCICE, and should
be clear to the developer before starting to modify and extend the source code
of it.

\backtocontents
 
% -----------------------------------------------------------------------------
 
\subsection{C++ programming concepts}
    
\begin{itemize*}
\item Classes
\item public, protected and private class scope modifiers
\item Derived classes by public inheritance
\item Abstract classes
\item Virtual methods
\item Pure virtual methods
\item Copy and assignment constructor
\item Memory allocation and deallocation with new and delete
\item The keyword const for methods and variables
\item Poiners and references
\item Generic programming with templates (classes and methods)
\item Template specialization
\item Typedefs
\item Enumerations
\item Static method variables
\item Namespaces
\item Definition, Declaration and Predeclaration of classes and methods
\item Shared pointers (smart pointers)
\end{itemize*}

\backtocontents
 
\subsection{Software Pattern}

\begin{itemize*}
\item Visitor
\item Factory method
\item Template method
\item Proxy
\end{itemize*}

\backtocontents
 
\subsection{Concepts taken from Peano (see Peano tutorials)}
 
\begin{itemize*}
\item Configuration concept
\item Unit and integration tests
\item Logging
\item Assertions
\end{itemize*}

\backtocontents

\subsection{Other used language features}

\begin{itemize*}
\item C++ standard template library containers (vector, map, list)
\item Boost C++ library (smart pointers, tuples, variant)
\item XML files
\end{itemize*}

\backtocontents
      
\section{Coding Rules and Style} \label{chap:styleguide}

The coding rules describe the set of rules on how to write source code and limit the
application of the C++ language features. The coding style describes the layout of
the source code. Both sets of rules are very important in a multi developer project
to create a high quality source code.

Most important: The following subsections do not give many examples of how the 
actual source code can look like. For examples check the source code of 
preCICE.

\backtocontents

\subsection{Rules for C++ language features}

Many of the rules are taken from the book "Effective C++" and "More Effective C++"
from Jack Meyers, Addison Wesley. These two books should be definitely read by
anyone who wants to program in C++. While the following rules do only describe the
"how" to code, the books do also explain why these rules make sense.
\begin{itemize*}
\item All sources are nested into the namespace "precice".
\item Sources which are not meant to be in the API of preCICE are nested into
further namespaces.
\item All class variables have to be made private, with the exception of static
const variables, which can be protected or public.
\item Functions returning a class variable do return const references to them, if
the variable is of user-defined type (objects). They return the variable by value
if it is of builtin type (int, double, char).
\item The variable modifiers \texttt{short}, \texttt{long} and \texttt{unsigned}
are used scarcly.
\item Builtin type parameters (int, double, char) are passed by-value to a
function. The \texttt{const} modifier is not used for pass-by-value parameters.
\item User-defined type parameters (objects) are passed as \texttt{const}
reference to functions, if their state is not intended to be changed, and as
reference parameters, if their state is changed by the function.
\item Member functions that do not alter the state of an Object are made const
\item Use of references is preferred over that of pointers.
\item  If a function returns one result, it should do so via a return value.
\begin{verbatim} 
   void computeResult ( int * result )
   {
     *result = /* computation */;  // WRONG!! Pass variable by return.
   }
   int computeResult ( void )
   {
     return /* computation */;   // RIGHT
   }
\end{verbatim}
\item If a function returns several results:
  \begin{itemize*}
  \item A \texttt{struct} or \texttt{std::tr1::tuple} can be returned holding the
  results.
  \item The results can be written into non-const reference parameters given to the
  function.
  \item Pointers can be used only for objects (not for built-in types), and only
  if the return value is created with the \texttt{new} operator inside of the
  function. The class defining the function must take care of deleting the memory
  allocated with \texttt{new} at some suitable point (Destructor, e.g.). A better
  way is to use smart pointers instead.
  \end{itemize*}
\item Non-virtual functions are not overwritten in subclasses.
\item Preprocessor macros are used only if it is not possible to use other
language features to achieve the same functionality.
\item Include guards are used to ensure unique includes of header (.hpp) files.
\item Pre-declarations of user-defined types should be used in header files, if
possible.
\end{itemize*}

\backtocontents

\subsection{Rules for code layout}
  
\begin{itemize*}
\item The source code is placed into files ending with .hpp and .cpp. The .hpp
files contain declarations, template and inline definitions. The .cpp files
contain all other definitions.
\item Coding and documentation is done in English.
\item No tabulators are used in the source code. Instead, only whitespaces are
used (Eclipse allows to configure this).
\item The indentation depth for code is 2 whitespaces per level.
\item The upper limit of characters (including whitespaces) in a line is 80
(Eclipse allows to display a vertical line at a given character number).
\item The hungarian naming convention for variables is \textbf{not} followed,
i.e. variables do not start with the description of its type (dVariable for
doubles, or iIndex for an int, e.g.). However, starting letters might be used
to signal special purposes of variables, such as loop counters starting with an
i.
\item Virtual function declarations in derived classes must always be enhanced
by the \texttt{virtual} keyword.
\item Preprocessor defines and macros are written in capital letters and start
with \texttt{PRECICE\_}. Words are separated by underscores
(\texttt{PRECICE\_TWO\_POWER\_DIM}, e.g.).
\item Variable names should give an idea about the meaning of the variable.
Abbreviations should be avoided. The same variable names should be used for the
same tasks. In precice, for loop variables are usually named \texttt{i},
\texttt{j}, \texttt{k} or \texttt{l}, iterator names start with \texttt{iter},
counters end with \texttt{Count} and so on.
\item Local Variables are defined as close to their point of use as possible.
\item Non-const variable names start with a lower case letter. They do not use
undersorces, but connect words directly with captial first letters
(\texttt{anExampleVariable}, e.g.).
\item Class member variable names start with an underscore
(\texttt{\_memberVariable}, e.g.).
\item Struct member variables do not start with underscore.
\item Names of userdefined types, i.e. classes, structs, unions, enums start
with Uppercase letters.
\item If a class is abstract and does contain no definitions, its name starts
with a capital \texttt{I} for interface.
\item If a class is abstract and does contain definitions, its name starts with
\texttt{Abstract}.
\item Method names start with a lower case letter.
\item The void keyword is not used to highlight empty method parameters.
\item Namespaces consist of one word, written in lower case letters.
\item Single (or few) line class methods can be defined directly in the
class' body.
\item A class defining virtual methods has to define a virtual destructor, even
if the destructor does nothing.
\end{itemize*}

\backtocontents

%   \section{Nightly builds and tests} \label{chap:nightlybuilds}
%   
%     In this section, nightly builds and test runs of preCICE are discussed and
%     explained. To automate these tasks, preCICE uses Buildbot
%     (\url{http://buildbot.net/trac}, User's guide:
%     \url{http://djmitche.github.com/buildbot/docs/0.7.10/#Introduction}). A
%     running version of a buildbot is located at the machine \texttt{xyz}, the
%     build and test results can be viewed with help of an internet browser at
%     \url{www.xyz.de}. Nontheless, a short guide on how to setup a buildbot
%     configuration and how to make it automately run is given here.
%     
%     As a prerequesit for buildbot, Twisted is needed (including the mail
%     package, which is not installed by default on the machines of the chair).
%     Twisted can be downloaded from \url{http://twistedmatrix.com/trac/}, where
%     the tarball version has to be chosen. Extract the files (you might have to
%     use the command \texttt{bunzip2} for extracting \texttt{.bz2}) and
%     change into the directory of the extracted files. Build the sources by
%     \\[2mm]
%     \verb|   $ python setup.py build |
%     \\[2mm]
%     After successful completion of the build, install the built files by using
%     the command
%     \\[2mm]
%     \verb|   $ python setup.py install --home=INSTALLATION_PATH |
%     \\[2mm]
%     In order to make buildbot recognize the installation location, the
%     environment variable \texttt{PYTHONPATH} needs to include the path to the
%     folder \texttt{INSTALLATION\_PATH/lib/python}.
%     
%     Then, download buildbot and perform the same actions as for the
%     installation of Twisted. If you do \textbf{not} choose the same installation
%     path, you have to add the path to the \texttt{PYTHONPATH} environment variable
%     again.
%     
%     To prepare for the running buildbot, a directory,holding all files
%     dedicated to one buildbot master, has to be created by using the command    
%     \\[2mm]
%     \verb|  $ buildbot create-master MASTER_DIRECTORY_NAME |
%     \\[2mm]
%     Within the created directory, a configuration file \texttt{master.cfg} has
%     to be created, which specifies all tasks to be done by the buildbot. The
%     configuration used for nightly builds with preCICE is contained in the
%     directory \texttt{PRECICE\_ROOT/tools/buildbot}. You can check the
%     configuration file by using the following command
%     \\[2mm]
%     \verb|  $ buildbot checkconfig CONFIG_FILENAME |
%     \\[2mm]
%     The buildbot executable runs as daemon, and has to be started by the command
%     \\[2mm]
%     \verb|  $ buildbot start MASTER_DIRECTORY_NAME | 
%     \\[2mm]
    
    
    

\end{document}
